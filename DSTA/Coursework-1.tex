\documentclass[12pt]{article}
\usepackage[utf8]{inputenc}
\usepackage[english]{babel}

\renewcommand{\baselinestretch}{1.3}

\usepackage{biblatex}
\addbibresource{cw1.bib}


\newcommand{\numpy}{{\tt numpy}}    % tt font for numpy

\topmargin -.5in
\textheight 9in
\oddsidemargin -.25in
\evensidemargin -.25in
\textwidth 7in

\begin{document}

% ========== Edit your name here
\author{Baran Buluttekin\\13153116}
\title{Data Science Techniques and Applications\\Coursework I}
\date{March 9, 2019}
\maketitle

\medskip

\indent For this coursework, I choose to examine Heart Disease UCI dataset \cite{uci-source} in kaggle \cite{kaggle}. It comprises of data that collected from patients from 4 medical institutions. There is no timestamp associated with data collections but according to UCI website where the kaggle dataset originally acquired, data donated at 1988. Its primarily used for classification machine learning tasks.

Dataset have total of 303 observations (rows) and 14 columns (features). There are 8 categorical columns present at the data but these variables represented in integers assigned to them. For example sex variable denotes male patients with 1 female patients with 0. Below is list of complete feature attributes\cite{uci-source}:
\begin{quote}
\begin{enumerate}
    \item age: Age in years (max:77, min:29)
    \item sex: Gender of the patient (0:Female, 1:Male)
    \item cp: Type of chest pain 
    \begin{enumerate}
        \item 1: indicate typical angina
        \item 2: indicate a typical angina 
        \item 3: indicate non-anginal pain 
        \item 4: indicate asymptomatic pain
    \end{enumerate}
    \item trestbps: Blood pressure recording for resting patient (measured in mm Hg)
    \item chol: cholesterol level (mg/dl) 
    \item fbs: Blood sugar level (fasting) $>$ 120 mg/dl (1:true, 0:false)
    \item restecg: Category for electrocardiographic, measured while patient resting
    \begin{enumerate}
        \item 0: indicate normal 
        \item 1: indicate abnormal ST-T wave mea    sure (T wave inversions and/or ST elevation or depression of $>$ 0.05 mV)
        \item 2: Estes criteria could be definite or probable ventricular hypertrophy
    \end{enumerate}
    \item thalach: Highest level of heart rate measurement
    \item exang: (0:no, 1:yes) Exercise induced angina 
    \item oldpeak: Measurement of ST depression
    \item slope: St segment slope value recorded at peak excersice
    \begin{enumerate}
        \item 1: indicate up-sloping 
        \item 2: indicate constant
        \item 3: indicate down-sloping
    \end{enumerate}
    \item ca: Fluoroscopy coloring have 4 category
    \item thal: Defect levels
    \item target: Outcome (1:Heart disease, 0:Normal)
\end{enumerate}
\end{quote}

For the PCA consideration most important dimensions are, age, chol (cholestoral) and thalach (maximum heart rate) because these dimensions have the highest variance which will be important for PCA. There is no indication of data integrity issues. For example age variable have values ranging from 29 to 77 which is expected and contains no abnormal values such as 0 or 250. Similar observation can be obtained from max heart rate measurement that is also have expected values according to general guidence \cite{maya}.

\printbibliography
\end{document}