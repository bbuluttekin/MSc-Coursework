\documentclass[12pt]{article}
\usepackage[utf8]{inputenc}
\usepackage[english]{babel}
 
\usepackage{biblatex}
\addbibresource{cw1.bib}

\newcommand{\numpy}{{\tt numpy}}    % tt font for numpy

\topmargin -.5in
\textheight 9in
\oddsidemargin -.25in
\evensidemargin -.25in
\textwidth 7in

\begin{document}

% ========== Edit your name here
\author{Baran Buluttekin}
\title{Data Science Techniques and Applications\\Coursework I}
\date{March 8, 2019}
\maketitle

\medskip
\textbf{Phase I}\\
\indent For this coursework after some searching I choose to examine Heart Disease UCI dataset in kaggle. It comprises of data that collected from patients from 4 medical institutions. There is no timestamp associated with data collections but according to UCI website data donated at 1988. Its primarily used for classification machine learning tasks.

Dataset have total of 303 observations (rows) and 14 columns (features). Below is list of complete feature attributes:
\begin{quote}
\begin{enumerate}
    \item age: Age in years (max:77, min:29)
    \item sex: Sex (1:Male, 0:Female)
    \item cp: Chest pain type
    \begin{enumerate}
        \item Value 1: typical angina
        \item Value 2: atypical angina 
        \item Value 3: non-anginal pain 
        \item Value 4: asymptomatic 
    \end{enumerate}
    \item trestbps: Resting blood pressure (in mm Hg on admission to the hospital)
    \item chol: Serum cholestoral in mg/dl 
    \item fbs: Fasting blood sugar $>$ 120 mg/dl (1:true, 0:false)
    \item restecg: Resting electrocardiographic results
    \begin{enumerate}
        \item Value 0: normal 
        \item   Value 1: having ST-T wave abnormality (T wave inversions and/or ST elevation or depression of $>$ 0.05 mV)
        \item Value 2: showing probable or definite left ventricular hypertrophy by Estes' criteria 
    \end{enumerate}
    \item thalach: Maximum heart rate achieved
    \item exang: Exercise induced angina (1:yes, 0:no)
    \item oldpeak: ST depression induced by exercise relative to rest
    \item slope: The slope of the peak exercise ST segment
    \begin{enumerate}
        \item Value 1: upsloping 
        \item Value 2: flat 
        \item Value 3: downsloping
    \end{enumerate}
    \item ca: number of major vessels (0-3) colored by flourosopy
    \item thal: 3:normal, 6:fixed defect, 7:reversable defect
    \item target: Outcome (1:Heart disease, 0:Normal)
\end{enumerate}
\end{quote}

\printbibliography
\end{document}