\documentclass[12pt]{article}
\usepackage{enumerate}

\begin{document}
    \begin{titlepage}
        \begin{center}
            \vspace*{.06\textheight}{\scshape\LARGE Birkbeck, University of London\par}\vspace{1.5cm} % University name
            \rule[0.5ex]{\linewidth}{2pt}\vspace*{-\baselineskip}\vspace*{3.2pt}
            \rule[0.5ex]{\linewidth}{1pt}\\[\baselineskip]
            %title of the report
            \huge{\bfseries Computer Systems\\Coursework1}\\[4mm]
            \rule[0.5ex]{\linewidth}{1pt}\vspace*{-\baselineskip}\vspace{3.2pt}
            \rule[0.5ex]{\linewidth}{2pt}\\
            [2.5cm]
        
            \textsc{\Large Baran Buluttekin}\\
            [2.5cm]
            \large \textit{ I have read and understood the sections of plagiarism in the College Policy on assessment offences and confirm that the work is my own, with the work of others clearly acknowledged. I give my permission to submit my report to the plagiarism testing database that the College is using and test it using plagiarism detection software, search engines or meta-searching software.}


        \end{center}
    \end{titlepage}
    \section*{Answers}
    \begin{enumerate}
        \item First
        \begin{enumerate}
            \item 1 LOAD r3, M\\
            2 LOAD r0, \#1\\
            3 LOAD r1, \#1\\
            4 LOAD r2, \#1\\
            5 SUB r3, r3, \#1\\
            6 ADD r4, r1, r0\\
            7 MUL r4, r4, r2\\
            8 LOAD r0, r1\\
            9 LOAD r1, r2\\
            10 LOAD r2, r4\\
            11 BNE 5, r3, \#3 // jump to instruction 5 if r3 not equal to 3\\
            12 STOR M, r2\\
            \\
            where \# indicates immediate addressing and BNE stands for "branch if not equal"
            \\
            \item other point
            \item The table \ref{table:1} is an example of referenced \LaTeX elements.
 
            \begin{table}[h!]
            \centering
            \begin{tabular}{||c||c c c c c c c||} 
             \hline
              & IF & ID & IW & RR & EX & WB & Comments\\ [0.5ex] 
             \hline\hline
             1 & 2 & 3 & 4 & 5 & 6 & 7 & This is a long explanation\\ 
             2 & 7 & 78 & 5415 & 5 & 6 & 7 &  \\
             3 & 545 & 778 & 7507 & 5 & 6 & 7 & \\
             4 & 545 & 18744 & 7560 & & & & \\
             5 & 88 & 788 & 6344 & & & & \\  
             6 & 88 & 788 & 6344 & & & & \\
             [1ex]
             \hline
            \end{tabular}
            \caption{Table to test captions and labels}
            \label{table:1}
            \end{table}
        \end{enumerate}
        \item 15 ns = 15$\times 10^{-9}$ seconds\\
        85 ns = 85$\times 10^{-9}$ seconds\\
        10 ms = 1 $\times 10^{-2}$ seconds\\
        \\
        Probability of being in main memory is 0.7 and cache hit ratio is 0.4. Therefore time to load is:\\
        0.7 $\times$ 1 $\times 10^{-2}$ + 0.3 $\times$ (0.4 $\times$ 15$\times 10^{-9}$ + 0.6 $\times$ 85$\times 10^{-9}$)

    \end{enumerate}

\end{document} 