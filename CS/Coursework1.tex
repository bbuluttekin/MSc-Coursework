\documentclass[12pt]{article}
\usepackage{enumerate}
\usepackage{longtable}

\begin{document}
    \begin{titlepage}
        \begin{center}
            \vspace*{.06\textheight}{\scshape\LARGE Birkbeck, University of London\par}\vspace{1.5cm} % University name
            \rule[0.5ex]{\linewidth}{2pt}\vspace*{-\baselineskip}\vspace*{3.2pt}
            \rule[0.5ex]{\linewidth}{1pt}\\[\baselineskip]
            %title of the report
            \huge{\bfseries Computer Systems\\Coursework Part 1}\\[4mm]
            \rule[0.5ex]{\linewidth}{1pt}\vspace*{-\baselineskip}\vspace{3.2pt}
            \rule[0.5ex]{\linewidth}{2pt}\\
            [2.5cm]
        
            \textsc{\Large Baran Buluttekin\\13153116}\\
            [1.5cm]
            \large \textit{ I have read and understood the sections of plagiarism in the College Policy on assessment offences and confirm that the work is my own, with the work of others clearly acknowledged. I give my permission to submit my report to the plagiarism testing database that the College is using and test it using plagiarism detection software, search engines or meta-searching software.}


        \end{center}
    \end{titlepage}
    \section*{Answers}
    \begin{enumerate}
        \item 
        \begin{enumerate}
            \item 1 LOAD r3, M\\
            2 LOAD r0, \#1  \hspace{1cm}// $f(n - 2)$\\
            3 LOAD r1, \#1 \hspace{1cm} // $f(n-1)$\\
            4 LOAD r2, \#1 \hspace{1cm} // $f(n)$\\
            5 SUB r3, r3, \#1\\
            6 ADD r4, r1, r0\\
            7 MUL r4, r4, r2\\
            8 LOAD r0, r1\\
            9 LOAD r1, r2\\
            10 LOAD r2, r4\\
            11 BNE 5, r3, \#3 \hspace{0.5cm} // jump to instruction 5 if r3 not equal to 3\\
            12 STOR M, r2\\
            \\
            where \# indicates immediate addressing and BNE stands for "branch if not equal"
            \\
            \item F (fetch) stage is first stage for all the instructions, where they loaded onto CPU. D (decode) is the stage where fetched instructions decoded. We assume data transfer between main memory and CPU happens E (execution) stage and using immediate addressing skips R (read) and E stages. For immediate addressing data is written back to register in W (write) stage. All arithmetic instructions will go through R, E stages and written back to register in W stage. If any instruction that can not move forward to further stages after the D stage for any reason, will be hold in IW (instruction window) until the restriction of the instruction is lifted. For example E stage is busy while next stage is execution or register instruction depends on is not ready for read.
            \clearpage
            \item Below is the pipeline for execution of program where k = 5
            \begin{table}[h!]
            \centering
            \begin{tabular}{||c||c|c|c|c|c|c|c||} 
             \hline
              & F & D & IW & R & E & W & Comments\\ [0.5ex] 
             \hline\hline
             1 & I1 & & & & & & \\ 
             2 & I2 & I1 & & & & &  \\
             3 & I3 & I2 & & & I1 & & I1 skips R\\
             4 & I4 & I3 & I2 & & & I1 & \\
             5 & I5 & I4 & I3 & & & I2 & I2, I3 skip R,E\\  
             6 & I6 & I5 & I4 & & & I3 & I4 skips R,E\\
             7 & I7 & I6 & & I5 & & I4 & \\
             8 & I8 & I7 & & I6 & I5 & & \\
             9 & I9 & I8 & I7 & & I6 & I5 & r3 = 4\\
             10 & I10 & I9 & I7, I8 & & & I6 & r4 not ready\\
             11 & I11 & I10 & I8, I9 & I7 & & & Read busy\\
             12 & I12 & I11 & I9, I10 & I8 & I7 & & Read busy for I9\\
             13 & & I12 & I11, I10 & I9 & & I7 & I8, I9, I10 skips E and W busy\\
             14 & & & I11, I12 & I10 & & I8 & R busy\\
             15 & & & I12 & I11 & & I9 & \\
             16 & & & I12 & \textbf{X} & I11 & I10 & Condition holds\\
             17 & & & I12 & \textbf{X} & & I11 & PC updated, I12 discarded\\
             18 & I5 & & & \textbf{X} & & & \\
             19 & I6 & I5 & & \textbf{X} & & & \\
             20 & I7 & I6 & & I5 & & & \\
             21 & I8 & I7 & & I6 & I5 & & \\
             22 & I9 & I8 & I7 & & I6 & I5 & r3 = 3\\
             23 & I10 & I9 & I7, I8 & & & I6 & r4 not ready\\
             24 & I11 & I10 & I8, I9 & I7 & & & Read busy\\
             25 & I12 & I11 & I9, I10 & I8 & I7 & & Read busy for I9\\
             26 & & I12 & I10, I11 & I9 & & I7 & I8, I9, I10 skips E and w busy\\
             27 & & & I11, I12 & I10 & & I8 & R busy\\
             28 & & & I12 & I11 & & I9 & \\
             29 & & & I12 & \textbf{X} & I11 & I10 & Condition fails\\
             30 & & & & I12 & & & No need to update PC\\
             31 & & & & & I12 & & \\
             [1ex]
             \hline
            \end{tabular}
            \caption{Table for pipeline execution for k = 5.}
            \label{table:1}
            \end{table}
        \end{enumerate}
        \item 15 ns = 15$\times 10^{-9}$ seconds\\
        85 ns = 85$\times 10^{-9}$ seconds\\
        10 ms = 1 $\times 10^{-2}$ seconds\\
        \\
        Probability of being in main memory is 0.7 and cache hit ratio is 0.4. Therefore average time to load is:\\
        $0.7 \times (1 \times 10^{-2} + 85 \times 10^{-9} + 15 \times 10^{-9}) + 0.3 \times (0.4 \times 15 \times 10^{-9} + 0.6 \times (85 \times 10^{-9} + 15 \times 10^{-9}))$\\
        $0,70000198 \times 10^{-2} seconds $\\
        $\approx 0,7 ms$

    \end{enumerate}

\end{document} 